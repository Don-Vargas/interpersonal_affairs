\documentclass[letterpaper,12pt]{article} % Define el tipo de documento (article) y establece el tamaño de papel (letter) y el tamaño de fuente (12pt)
\usepackage[utf8]{inputenc} % Permite la entrada de caracteres en codificación UTF-8, útil para acentos y caracteres especiales
\usepackage[spanish]{babel} % Configura el documento para que use las convenciones del idioma español (traducciones, nomenclatura, etc.)
\usepackage{setspace} % Permite configurar el interlineado del documento
\usepackage{geometry} % Proporciona una forma sencilla de establecer márgenes y dimensiones de la página
\usepackage{parskip} % Gestiona el espaciado entre párrafos, eliminando la sangría en el primer párrafo y añadiendo espacio entre párrafos
\usepackage{mathptmx} % Establece la fuente a una alternativa cercana a Times New Roman

% Configuración de márgenes
\geometry{ % Comienza la configuración de márgenes
	left=2.5cm, % Establece el margen izquierdo a 2.5 cm
	right=2.5cm, % Establece el margen derecho a 2.5 cm
	top=2.5cm, % Establece el margen superior a 2.5 cm
	bottom=2.5cm % Establece el margen inferior a 2.5 cm
}

% Configuración de la fuente
% La fuente se establece con el paquete mathptmx, por lo que no se requiere una configuración adicional

\setstretch{2.0} % Configura el interlineado a 2.0, cumpliendo con las normas APA

% Configuración de la sangría
\setlength{\parindent}{5em} % Establece la sangría en la primera línea de cada párrafo a 5 espacios (5em)

% Configuración de espaciado entre párrafos
\setlength{\parskip}{0pt} % Elimina el espacio adicional entre párrafos, cumpliendo con las normas APA


\title{Análisis y mejora de relaciones interpersonales aplicando SCRUM como herramienta de gestión}
\author{Héctor Alejandro Vargas Gutiérrez}
\date{\today}

\begin{document}
	
	\maketitle
	
	\tableofcontents % Genera el índice
	\newpage % Empieza una nueva página
	
	\section{Introducción}
	\subsection{Planteamiento del problema}
	%Contexto general sobre el tema.
	%Contexto: Proporcionar información general sobre el tema de estudio. Describir el área de %investigación y su importancia.
	
	%Situación actual del área de estudio.
	%Situación Actual: Presentar el estado actual del conocimiento o práctica en el área. Esto puede %incluir estadísticas, antecedentes y referencias a trabajos anteriores.
	
	Dificultades en la gestión y mejora de relaciones interpersonales.
	
	\subsection{Descripción del Problema}
	%Identificación del problema específico
	%Identificación: Especificar claramente el problema que se va a investigar. ¿Qué es lo que se necesita resolver o entender mejor?
	
	%Consecuencias del problema.
	%Consecuencias: Explicar las implicaciones del problema, es decir, cómo afecta a la comunidad, a la disciplina o al sector involucrado
	
	\subsection{Justificación}
	
	% Importancia de abordar el problema
	%Relevancia: Argumentar por qué es importante abordar este problema. Esto puede incluir aspectos sociales, económicos, académicos o prácticos.
	
	%Cómo la investigación contribuirá al área.
	%Contribución: Indicar cómo la investigación contribuirá al conocimiento existente o a la práctica en el campo.
	
	Importancia de un enfoque estructurado en la gestión de relaciones.
		
	\subsection{Objetivos}
	%Objetivo general.
	%Objetivo General: Definir el propósito principal de la investigación en relación con el problema planteado.
	
	%Objetivos específicos.
	%Objetivos Específicos: Desglosar el objetivo general en metas más concretas y medibles.
	
	Aplicar SCRUM y metodologías complementarias para la mejora continua de relaciones personales.
	
	\subsection{Preguntas de Investigación}
	%Preguntas formuladas en relación con el problema y los objetivos.
	%Formulación de Preguntas: Plantear preguntas específicas que la investigación busca responder. %Estas deben estar alineadas con el problema y los objetivos.
	
	\subsection{Delimitación del Estudio}
	%Alcance y limitaciones de la investigación.
	%Alcance: Definir los límites de la investigación, incluyendo la población, el contexto y los aspectos que se abordarán y los que se excluirán.
	
	\section{Marco Teórico}
	\subsection{Tipologías de relaciones}
	Familia, amistades, relaciones de pareja.
	
	\subsection{Teorías sobre la gestión de relaciones.}
	
	\subsection{Introducción a SCRUM y su adaptación a contextos interpersonales.}
	
	\subsection{Círculos de Diálogo Estructurado y Método de Reuniones de Check-in Semanal o Mensual}
	Descripción de ambas metodologías y su relevancia para la comunicación efectiva y la revisión periódica de relaciones.
	
	\section{Metodología SCRUM aplicada a las relaciones interpersonales}
	\subsection{Roles, eventos y artefactos SCRUM adaptados a las relaciones personales}
	Definición de roles: Product Owner (quien establece metas), Scrum Master (quien facilita las discusiones) y Equipo de Desarrollo (la relación misma).
	
	\subsection{Frecuencia variable para reuniones}
	Las "daily meetings" de SCRUM se adaptan a una frecuencia semanal o mensual, según las necesidades de la relación, permitiendo así un enfoque más flexible y realista.
	
	\section{Círculos de Diálogo Estructurado}
	\subsection{Implementación de círculos de diálogo en las interacciones}
	Revisión de necesidades, logros y retos de manera estructurada. Establecimiento de un espacio seguro para la comunicación y la resolución de conflictos.
	
	\section{Método de Reuniones de Check-in Semanal o Mensual}
	\subsection{Estructura de las reuniones}
	\begin{itemize}
		\item Revisión de necesidades actuales.
		\item Logros recientes.
		\item Dificultades.
		\item Retos futuros.
	\end{itemize}
	
	\subsection{Integración con SCRUM}
	Estas reuniones se combinan con los eventos de SCRUM, permitiendo una revisión más profunda en momentos clave, alineando objetivos y facilitando el crecimiento de la relación.
	
	\section{Preguntas claves para cada fase de SCRUM}
	\subsection{Planificación, revisión y retrospectiva}
	Definición de preguntas que guíen la evaluación y mejora continua de la relación, adaptadas a la frecuencia de las reuniones.
	
	\section{Adaptación flexible de roles SCRUM}
	\subsection{Rotación de roles según la situación en relaciones personales}
	Cómo cada persona puede asumir diferentes roles en distintas etapas de la relación.
	
	\section{Motivo de omisión de relaciones laborales}
	Se omiten las relaciones laborales de manera puntual y específica porque la metodología de referencia (SCRUM) se toma de este contexto, donde la dinámica y objetivos suelen ser diferentes a los de las relaciones personales. Las interacciones laborales, aunque importantes, pueden estar regidas por estructuras y objetivos organizacionales que no se alinean con la flexibilidad y la intimidad requeridas en relaciones más personales.
	
	\section{Conclusiones y recomendaciones}
	
	este es un ejemplo de bibliografía en formato apa: \cite{test0} 
	
	sdfsdf
	\cite{test1}jejeje saludos.	
	
	
	Resumen de los hallazgos y sugerencias para la implementación de las metodologías en la vida diaria.
	
	
	
	\bibliographystyle{apalike}
	\bibliography{referencias}
\end{document}
